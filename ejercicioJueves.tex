\documentclass[10pt,a4paper]{book}
\usepackage[utf8]{inputenc}
\usepackage[spanish]{babel}
\usepackage{amsmath}
\usepackage{amsfonts}
\usepackage{amssymb}
\usepackage{makeidx}
\usepackage{graphicx}
\usepackage{lmodern}
%\usepackage{subfigure}
%\usepackage{wrapfig}
\usepackage[left=2cm,right=2cm,top=2cm,bottom=2cm]{geometry}
\author{María del Carmen González Martínez}
\title{{\Huge Ejercicio del clase} }
\date{\today}
\begin{document}
\maketitle


\tableofcontents

%nueva página en blanco sin título
\newpage
\section{Listas}
\textbf{\textit{Listas con item}}
\begin{itemize}
\item Item1. En este caso ...
\item Item2. En este otro caso ...
\item Item3. En este caso ...
\end{itemize}


\textbf{\textit{Listas enumeradas}}
\begin{enumerate}
\item Partimos la cebolla en juliana.
\item La sofreimos a fuego lento, hasta que esté transparente.
\item Añadimos el bacon troceado a la cebolla doradita.
\item Añadimos la nata y removemos hasta que espese.
\item Finalmente añadimos el preparado a los macarrones. 
\end{enumerate}

\textbf{\textit{Listas para la compra}}
\begin{itemize}

\item[1] Cebolla

\item[2] Bacon 
\begin{itemize} 
	\item Bacon ahumado \item  Bacon salado
\end{itemize}

\item[3] Manzanas
  \begin{itemize}
	\item[3.1] Manzana Golden
	\item[3.2] Manzana Reineta
    \item[3.3] Manzana Arinosa
  \end{itemize}

\item[4] Aceite de oliva
\\
\end{itemize}



%nueva página en blanco sin título
\newpage
\section{Ecuaciones}


{\LARGE Fórmulas matemáticas \\}
\begin{equation} 
\oint_{C} \overrightarrow{E} \cdot \overrightarrow{dl} = - \frac{d}{dt} \int_{S} \overrightarrow{B \cdot d\overrightarrow{S}} \ \ (Ley\ de\ Faraday)
\end{equation}

\mbox{este texto NO está enmarcado}
\fbox{pero este texto está enmarcado}


%nueva página en blanco sin título
\newpage
\section{Imágenes en latex}


\begin{center}
\textbf{{\Huge \textit{Dibujos animados y pelis \\}}}
\end{center}

\textbf{{\Large \textit{Imagen en escala 0.8}}}
\begin{figure}[htb]
\centering
\includegraphics[scale=1, angle=45]{../../Desktop/minions.jpeg} 
\caption{Minions}\label{fig:Virtual}
\end{figure}


\textbf{{\Large \textit{Figura compuesta}}}
\begin{figure}
\centering
%\subfigure[Starks]{}
\includegraphics[scale=0.5]{../../Desktop/mounstrosSA.jpeg}
\caption{Mounstros SA}\label{fig:mounstros} 
%\subfigure[Starks]{\includegraphics[scale=0.5]{../Directorio/Estudios/DOCTORADO-TESIS/imagenes/espacio holografico.jpeg}}
\includegraphics[scale=0.5]{../../Desktop/nemoDori.jpeg} 
\caption{Buscando a nemo}\label{fig:nemo}
\end{figure}

%\textbf{{\Large \textit{Figura y texto}}}
%\begin{wrapfigure}{r}{40mm}
%\begin{center}
%\subfigure[Starks]{}
%\includegraphics[scale=1]{../../Desktop/perrillo.jpeg} 
%\subfigure[Starks]{\includegraphics[scale=0.5]{../Directorio/Estudios/DOCTORADO-TESIS/imagenes/espacio holografico.jpeg}}
%\paragraph{Este texto debería de salir al lado}
%\end{center}
%\end{wrapfigure}


\begin{figure}[h]
\begin{center}
\begin{tabular}{lll}
\includegraphics[scale=0.3]{../../Desktop/minions.jpeg}
& \includegraphics*[scale=0.3]{../../Desktop/minions.jpeg}
& \includegraphics*[scale=0.3]{../../Desktop/minions.jpeg}\\

\end{tabular}
\end{center}
\caption{Figuras en formato...}\label{ML:figuras262728}
\end{figure}







\end{document}