\documentclass[10pt,a4paper]{article}
\usepackage[utf8]{inputenc}
\usepackage[spanish]{babel}
\usepackage{amsmath}

\usepackage{amsfonts}
\usepackage{amssymb}

\author{Maria del Carmen}
\begin{document}

%ejemplos de ecuaciones sencillas puntuales
$ a =  b  *  x  +  c $ % sin formato centrado...	
$$ a =  b  *  x  +  c $$ % centrado, destacado
\\ \\
Tenemos la equivalencia $\frac{a}{b}= \frac{c}{d}$, válida para todo $a$, $b$, $c$, $d$ \\ \\
Tenemos la equivalencia $$\frac{a}{b}=\frac{c}{d}$$ válida para todo  $a$, $b$, $c$, $d$ \\ \\


%si queremos escribir mas ecuaciones definimos un entorno
\begin{displaymath}
  y=x^{2} * \pi 
\end{displaymath}\\ 

\begin{center}
\textbf{Ecuaciones de Maxwell:} \\
\end{center} 

\begin{equation}
\phi = \oint_{S} \overrightarrow{E}\cdot d\overrightarrow{S}= \frac{q_{enc}}{\varepsilon_{0}}   (Ley\ de\ Gauss)
\end{equation} 

\begin{equation}
\oint_{S} \overrightarrow{B}\cdot d\overrightarrow{S} = 0(Ley\ de\ Gauss\ para\ el\ campo\ matematico
\end{equation}

\begin{equation} 
\oint_{C} \overrightarrow{E} \cdot \overrightarrow{dl} = - \frac{d}{dt} \int_{S} \overrightarrow{B \cdot d\overrightarrow{S}} \ \ (Ley\ de\ Faraday)
\end{equation}


\end{document}